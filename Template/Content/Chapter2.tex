\chapter{Οδηγίες για τη Μορφή της Διατριβής}
\label{ch:Instructions}

Η μορφή αυτή καθιερώθηκε το 2005 και ενημερώθηκε το 2016. 
Από το 2019 ενημερώθηκε εκ νέου και προσαρμόστηκε ώστε να καλύπτει το νέο 
πρόγραμμα ``Μηχανική Δεδομένων και Υπολογιστικών συστημάτων''.
Στο κείμενο αυτό περιγράφεται η μορφή της διατριβής και παρέχονται σύντομες οδηγίες για τη χρήση του προτύπου Latex.

Προτείνεται η χρήση του Latex για την μορφοποίηση του κειμένου και παρέχεται πλήρες πρότυπο που παράγει αυτόματα κείμενο στη σωστή μορφή.
Βρίσκεται στο \url{https://github.com/vvdimako/cseuoi-thesis} σε δύο παραλλαγές: μία για το παλιό μεταπτυχιακό πρόγραμμα Πληροφορικής και μία για το νέο πρόγραμμα στη Μηχανική Δεδομένων και Υπολογιστικών Συστημάτων.

Μπορούν βέβαια να χρησιμοποιηθούν άλλες εφαρμογές στοιχειοθεσίας κειμένου (Word, Open-Libre Office, κλπ), αλλά δεν διατίθεται πλήρες πρότυπο για αυτές, παρά μόνο ένα απλό βοήθημα το οποίο έχει σημαντικές ατέλειες, δεν ακολουθεί πλήρως τις απαιτήσεις μορφοποίησης και χρειάζεται πολλές, «χειροκίνητες» αλλαγές για να μπορέσει το τελικό κείμενο να έρθει στη σωστή μορφή.
Είναι ευθύνη του Συγγραφέα να κάνει τις απαιτούμενες αλλαγές ώστε το τελικό κείμενο να ακολουθεί το πρότυπο μορφοποίησης του Τμήματος. 


\section{Διαδικασία Υποβολής Τελικού Αντίτυπου Διατριβής}
\label{sec:Submission}
Για τη διαδικασία υποβολής, ανατρέξτε στον κανονισμό και σε πληροφορίες στην 
ιστοσελίδα του Τμήματος.
(Οι οδηγίες αφαιρέθηκαν από το κείμενο αυτό για να μην υπάρχει η πληροφορία 
σε πολλά σημεία γιατί με τις τροποποιήσεις που συμβαίνουν είναι πιθανό να 
υπάρξουν διαφορές μεταξύ τους.)


\section{Διαμόρφωση Κειμένου}
\label{sec:Text}

\subsection{Βασικές Οδηγίες}
\label{subsec:Basic}


Η διατριβή πρέπει να είναι τυπωμένη σε μονή όψη, ενώ το κείμενο πρέπει να είναι εντός των εξής περιθωρίων (ίδια και στις μονές και στις ζυγές σελίδες):
\begin{itemize}
	\item Top: 2.5 cm.
	\item Bottom: 3 cm.
	\item Left: 2.5 cm.
	\item Right: 2.5 cm.
\end{itemize}

Σε όλη τη διατριβή, εκτός από τα Σχήματα και τους Πίνακες (συμπεριλαμβανομένων των λεζάντων όμως), πρέπει να χρησιμοποιείται γραμματοσειρά GFS Didot για το κανονικό κείμενο και Ubuntu Mono για εντολές (ως typewriter font), μεγέθους 12 στιγμών.

Οι παράγραφοι θα πρέπει να είναι πλήρως στοιχισμένες (justified) και με διάστημα μεταξύ γραμμών (διάστιχο, line spacing) στις 1.4 γραμμές.
Η εσοχή της πρώτης γραμμής της παραγράφου είναι 20 στιγμές (ή 0.7 cm), \textbf{αλλά η πρώτη παράγραφος ενός κεφαλαίου, ενότητας, κλπ, δεν έχει εσοχή}.
Δεν υπάρχει επιπλέον κενό διάστημα πρίν ή μετά από κάθε παράγραφο (spacing before, after).

Στο Word πρέπει να χρησιμοποιείται το στυλ Body Text που έχει τις παραπάνω ιδιότητες.
Μόνο για την πρώτη παραγράφο, που δεν έχει εσοχή, θα πρέπει να τροποποιούνται οι ιδιοτητες παραγράφου (indentation, first line 0).
Επιπλέον χρειάζεται ιδιαίτερη προσοχή όταν προστίθεται ή μετακινείται κείμενο ώστε να διατηρηθεί η σωστή μορφοποίηση των παραγράφων.

Οι όροι μπορούν να είναι σε \textit{πλάγια γράμματα}, 
χρησιμοποιώντας την εντολή \verb|\textit{πλάγια γράμματα}|, συνήθως την 
πρώτη φορά που χρησιμοποιούνται. Τα \textbf{έντο\-να γράμματα} πρέπει να 
χρησιμοποιούνται μόνο στην περίπτωση που είναι απαραίτητα για την κατανόηση 
του κειμένου, χρησιμοποιώντας την εντολή\\
\verb|\textbf{έντονα γράμματα}|.

Αν θέλετε να εισάγετε έναν νέο όρο στο ευρετήριο, θα πρέπει να χρησιμοποιήσετε 
την εντολή\index{νέος όρος} \verb|\index{νέος όρος}|. Για κάθε υποόρο που 
θέλετε να προστεθεί σε έναν προηγούμενο όρο, θα πρέπει να χρησιμοποιήσετε την 
εντολή\index{νέος όρος!υποόρος}\\
\verb|\index{νέος όρος!υποόρος}|.

Όταν θεωρείτε ότι είναι απολύτως απαραίτητο, μπορείτε να εισάγετε μία 
υποσημείωση με την εντολή \verb|\footnote{Αυτή είναι μία υποσημείωση.}|, 
η οποία εμφανίζεται στο κάτω μέρος της αντίστοιχης 
σελίδας\footnote{Αυτή είναι μία υποσημείωση.}.

To πακέτο \texttt{xgreek}\footnote{\url{https://www.ctan.org/pkg/xgreek?lang=en}.} 
ορίζει διάφορες χρήσιμες μακροεντολές, οι οποίες επιτρέπουν την εύκολη χρήση 
χαρακτήρων που είναι δύσκολα προσβάσιμοι από το πληκτρολόγιο.
Μερικές από αυτές τις μακροεντολές είναι οι εξής:
\begin{itemize}
	\item Άνω τόνος (\anwtonos): \verb|\anwtonos|.
	\item Άνω τελεία (\anoteleia): \verb|\anoteleia|.
	\item Σύμβολο του Ευρώ (\euro): \verb|\euro|.
	\item Σύμβολο τοις χιλίοις (\permill): \verb|\permill|.
\end{itemize}

Το κύριο αρχείο είναι το \texttt{SampleThesis.tex}, στο προοίμιο του οποίου 
θα πρέπει να περάσετε στο πακέτο \texttt{cseuoi-thesis} τις κατάλληλες 
επιλογές για τη διατριβή σας. Αν το κείμενο της διατριβής είναι στα Ελληνικά 
τότε θα πρέπει να περάσετε την επιλογή \texttt{gr}, ενώ αν το κείμενο της 
διατριβής είναι στα Αγγλικά τότε θα πρέπει να περάσετε την επιλογή \texttt{en}.
Για τη στοιχειοθεσία Μεταπτυχιακής Διπλωματικής Εργασίας θα πρέπει να περάσετε 
την επιλογή \texttt{msc}, ενώ για τη στοιχειοθεσία Διδακτορικής Διατριβής 
θα πρέπει να περάσετε την επιλογή \texttt{phd}.
Στην περίπτωση της Διδακτορικής Διατριβής δεν περνάτε κάποια άλλη επιλογή, ενώ
στην περίπτωση της Μεταπτυχιακής Διπλωματικής Εργασίας θα πρέπει επιπλέον να περάσετε 
την κατάλληλη επιλογή για μία από τις εξής ειδικεύσεις:
\begin{itemize}
	\item Προηγμένα Υπολογιστικά Συστήματα: \texttt{systems}.
	\item Επιστήμη και Μηχανική Δεδομένων: \texttt{data}.
\end{itemize}

Στη συνέχεια του ίδιου αρχείου θα πρέπει να συμπληρώσετε τα στοιχεία σας 
στις αντίστοιχες εντολές, αφαιρώντας την εντολή \verb|\colorbox{gray}{}|.
Ειδικότερα, τα στοιχεία που θα πρέπει να συμπληρώσετε είναι ο τίτλος της 
διατριβής, το ονοματεπώνυμο του φοιτητή, ο μήνας και το έτος αποφοίτησης, 
καθώς και το ονοματεπώνυμο και τη βαθμίδα του επιβλέποντος καθηγητή.
Τα παραπάνω στοιχεία θα πρέπει να τα συμπληρώσετε και στα Ελληνικά και στα 
Αγγλικά προκειμένου να χρησιμοποιηθούν κατάλληλα σε διάφορα σημεία της 
διατριβής, όπως η σελίδα τίτλου και οι σελίδες με τις περιλήψεις.
Αν κάποια τμήματα της διατριβής σας είναι σκιασμένα, είτε δεν έχετε 
συμπληρώσει τα αντίστοιχα στοιχεία σας είτε δεν αφαιρέσατε την εντολή 
\verb|\colorbox{gray}{}| όταν τα συμπληρώσατε.

 Η σειρά των προκαταρκτικών τμημάτων: σελίδα τίτλου - εξώφυλλο,
εξεταστική επιτροπή, αφιέρωση, ευχαριστίες, % περιεχόμενα, κατάλογος σχημάτων, κατάλογος πινάκων, κατάλογος αλγορίθμων, γλωσσάρι, περίληψη, extended abstract,
κλπ. θα πρέπει να είναι όπως στο παρόν έγγραφο.
 Στη συνέχεια ακολουθεί το κυρίως κείμενο, οργανωμένο σε κεφάλαια και μετά τυχόν παραρτήματα,
ευρετήριο όρων, δημοσιεύεσεις συγγραφέα και σύντομο βιογραφικό σημείωμα του συγγραφέα.

\subsection{Αρίθμηση Σελίδων}
\label{subsec:pageNumbering}

 Οι σελίδες πρίν τα «Περιεχόμενα» δεν αριθμούνται.
 Η αρίθμηση σελίδων από τα περιεχόμενα μέχρι και την σελίδα πρίν το πρώτο κεφάλαιο γίνεται με λατινική αρίθμηση (i, ii, κλπ).
 Στην συνέχεια η αρίθμηση ξεκινά από την αρχή και γίνεται με τη συνηθισμένη αραβική αρίθμηση (1, 2, κλπ), μέχρι και το τέλος του ευρετηρίου.
 Σημειώνεται ότι οι αριθμοί σελίδων, όπου υπάρχουν, βρίσκονται στο κάτω μέρος της σελίδας και στο κέντρο του κειμένου.

 Για να επιτευχθεί αυτό στο Word, χρειάζεται να χωριστεί το κείμενο σε Sections με διαφορετικές ιδιότητες.
 Έχει τοποθετηθεί προειδοποιητικό κείμενο (μόνο στο υπόδειγμα Word) σε αυτά τα σημεία, που φυσικά θα πρέπει να αφαιρεθεί από το τελικό κείμενο.


\subsection{Κεφάλαια, Ενότητες κλπ.}
\label{subsec:Chapters}

Για να ξεκινήσετε ένα κεφάλαιο, χρησιμοποιείτε την εντολή 
\verb|\chapter{Τίτλος Κεφαλαίου}|. Παρόμοια, για μία ενότητα 
χρησιμοποιείτε την εντολή \verb|\section{Τίτλος Ενότητας}|, ενώ για υποενότητα 
την εντολή \verb|\subsection{Τίτλος Υποενότητας}|.
Είναι επιθυμητό να εισάγετε και μία ετικέτα κάθε φορά που χρησιμοποιείτε 
τις παραπάνω εντολές, το οποίο γίνεται με την εντολή \verb|\label{Ετικέτα}|, 
προκειμένου να μπορείτε να αναφέρεστε στο αντίστοιχο σημείο του κειμένου 
με την εντολή \verb|\ref{Ετικέτα}|.

  Στο Word χρησιμοποιείστε το στυλ \texttt{Heading 1} για τίτλο κεφαλαίου (προσθέτωντας ένα page break πριν για να εξασφαλίσετε ότι το κεφάλαιο ξεκινά από νέα σελίδα), \texttt{Heading 2} για ενότητα και \texttt{Heading 3} για υποενότητα.
  Μπορεί, κατά περίπτωση, να χρειαστεί να αυξήσετε την απόσταση από το υπόλοιπο κείμενο σε (υπο)ενότητες αλλάζοντας τις ιδιότητες παραγράφου spacing before - after.

Η μορφοποίηση του τίτλου του κεφαλαίου έχει ώς εξής:
Το κεφάλαιο ξεκινάει σε νέα σελίδα και από το επάνω μέρος της, επιπλέον του περιθωρίου, υπάρχει διάστημα 90 στιγμών και μετά η λέξη Κεφάλαιο και ο αριθμός κεφαλαίου, στοιχισμένα αριστερά, σε έντονα γράμματα τύπου small caps\footnote{Όλα γράμματα είναι κεφαλαία, αλλά τα ``πραγματικά κεφαλαία᾽᾽ π.χ. το 1ο γράμμα, είναι μεγαλύτερα.}, μεγέθους 18 στιγμών.
Ακολουθεί διάστημα 35 στιγμών και ο τίτλος του κεφαλαίου στοιχισμένος δεξιά, σε έντονα γράμματα τύπου small caps, μεγέθους 18 στιγμών.
Από κάτω υπάρχει διάστημα 55 στιγμών και μετά ο μίνι πίνακας περιεχομένων του κεφαλαίου, που περιγράφεται παρακάτω.

Οι τίτλοι κεφαλαίων, ενοτήτων, και οι λεζάντες σχημάτων κλπ, έχουν κεφαλαία γράμματα το πρώτο του τίτλου και μετά μόνο στα πρώτα γράμματα των Ουσιαστικών, Ρημάτων και Επιθέτων.
Αυτό πρέπει να γίνεται χειροκίνητα ακόμη και σε κείμενο γραμμένο σε Latex.
Στο Word, αφαιρέστε τους τόνους των κεφαλαίων γραμμάτων, όταν είναι στη μέση της λέξης, γιατί φαίνονται άσχημα, αλλά θα λείπουν στα περιεχόμενα και θα χρειαστεί, με κάποιο τρόπο, να ξαναμπούν.

Ο μίνι πίνακας περιεχομένων ξεκινάει και τελειώνει με οριζόντιες γραμμές.
Αν δεν υπάρχουν ενότητες στο κεφάλαιο, τότε τοποθετείται μόνο μία οριζόντια γραμμή (για παράδειγμα βλ. Παράρτημα \ref{app:FirstAppendix}).
Στον πίνακα περιλαμβάνονται μόνο οι ενότητες πρώτου επιπέδου (π.χ. 2.1, 2.2, αλλά όχι 2.1.1) με την αριθμησή τους και με έντονους χαρακτήρες 10 στιγμών.
Δεν δίνονται οι αριθμοί σελίδας της κάθε ενότητας στον μίνι πίνακα.
Στο Latex ο μίνι πίνακας περιεχομένων δημιουργείται αυτόματα, αλλά στο Word πρέπει να γραφτεί με το χέρι, με πίνακα ή με μια απλή γραμμή.

Ο τίτλος των προκαταρκτικών τμημάτων (Αφιέρωση, κλπ), της Βιβλιογραφίας, Ευρετηρίου, κ.α. μορφοποιείται επίσης με έντονους χαρακτήρες, small caps 18 στιγμών,  αλλά με στοίχιση στα αριστερά και με μια οριζόντια γραμμή από κάτω.
Στο Word χρησιμοποιούνται μερικά διαφορετικά είδη στυλ που εξηγούνται στην ενότητα \ref{subsec:Contents}.


Οι τίτλοι ενοτήτων πρώτου και δεύτερου επιπέδου αριθμούνται με τον αριθμό κεφαλαίου, αριθμό ενότητας, αριθμό υποενότητας, με τελείες ανάμεσα στους αριθμούς και μορφοποιούνται με έντονους χαρακτήρες 14 στιγμών.

Ενότητες τρίτου επιπέδου καλό είναι να αποφεύγονται.
Δεν έχουν αρίθμηση, δεν εμφανίζονται στα περιεχόμενα, και μορφοποιούνται με έντονους χαρακτήρες 12 στιγμών.


\subsection{Παραρτήματα}
\label{subsec:Appendices}
Προαιρετικά, μπορείτε να εισάγετε ένα ή περισσότερα παραρτήματα, τα οποία 
θα βρίσκονται μετά τη βιβλιογραφία και πριν το ευρετήριο.
Η μορφοποίησή τους είναι όμοια με τα κεφάλαια με τη διαφορά ότι η αρίθμηση των παραρτημάτων 
γίνεται με κεφαλαίους Ελληνικούς χαρακτήρες.

Ειδικά στο Word χρησιμοποιούνται διαφορετικά στύλ για να γίνει σωστά η αρίθμησή τους.
Το Heading 6 αντιστοιχεί στον τίτλο παραρτήματος και τα Heading 7, Heading 8 στα δύο πρώτα επίπεδα ενοτήτων. 
H αρίθμηση γίνεται με λατινικά γράμματα και όχι με Ελληνικά και θα πρέπει να αλλαχθούν και στους τίτλους και στις λεζάντες και στους αντίστοιχους καταλόγους και τα περιεχόμενα.

\section{Σχήματα}
\label{sec:Figures}
Τα σχήματα και οι λεζάντες τους πρέπει να είναι πάντα κεντραρισμένα και εντός 
των περιθωρίων του κειμένου. Για να εισάγουμε ένα σχήμα στο κείμενο, 
χρησιμοποιούμε τις παρακάτω εντολές:

\begin{verbatim}
\begin{figure}[t]
 \centering
 \includegraphics[width=0.65\textwidth]{Figures/ExponentialFunction.pdf}
 \caption{Η Εκθετική Συνάρτηση.}
 \label{fig:ExponentialFunction}
\end{figure}
\end{verbatim}

\begin{figure}[t]
	\centering
	\includegraphics[width=0.65\textwidth]{Figures/ExponentialFunction.pdf}
	\caption{Η Εκθετική Συνάρτηση.}
	\label{fig:ExponentialFunction}
\end{figure}

Όλα τα σχήματα, πίνακες, κλπ. πρέπει να είναι αριθμημένα και να αναφέρονται στον αντίστοιχο Κατάλογο Σχημάτων, Πινάκων, κλπ.
Η αρίθμησή τους ξεκινά από το 1 σε κάθε νέο κεφάλαιο.
Ο αριθμός σχήματος θα πρέπει να περιλαμβάνει τον αριθμό κεφαλαίου, και τον αριθμό σχήματος με μία τελεία μεταξύ τους.

Η αναφορά σε ένα σχήμα γίνεται με τον εξής τρόπο: 
«το αποτέλεσμα των παραπάνω εντολών φαίνεται στο 
Σχήμα~\ref{fig:ExponentialFunction}».
Προτείνεται η χρήση labels και cross-references αντί για αρίθμηση με το χέρι γιατί έτσι οι αλλαγές στη σειρά σχημάτων αλλάζουν αυτόματα και την αρίθμηση.
Στο Latex, χρησιμοποιούμε την εντολή 
\verb|Σχήμα~\ref{fig:ExponentialFunction}|.
Στο Word εισάγετε cross-reference και βρίσκετε το αντίστοιχο σχήμα.
Επειδή θα υπάρχει και η λέξη "Σχήμα" που συχνά δεν εξυπηρετεί (π.χ. αν θέλετε να γράψετε «του Σχήματος»), θα πρέπει να αλλάξετε τον τρόπο εμφάνισης του αριθμού σχήματος: Δείτε την κωδικοποίηση του (field code) με τα πλήκτρα Alt-F9, και προσθέστε \verb|\# "x.0x"| μεταξύ του κωδικού \verb|_Refxxxx| και του \verb|\h|.
Συνήθως χρειάζεται να κάνετε Update Field (με δεξί κλικ στο γκρί field) για να δείτε την αλλαγή.

Αν θέλουμε να εισάγουμε σε ένα σχήμα πολλές εικόνες μαζί, τότε χρησιμοποιούμε 
τις παρακάτω εντολές:

\begin{verbatim}
\begin{figure}[t]
 \centering
 \begin{subfigure}[t]{0.3\textwidth}
  \centering
  \includegraphics[height=0.15\textheight]{Figures/GraphA.pdf}
  \caption{}
  \label{subfig:GraphA}
 \end{subfigure}
 \hfill
 \begin{subfigure}[t]{0.3\textwidth}
  \centering
  \includegraphics[height=0.15\textheight]{Figures/GraphB.pdf}
  \caption{}
  \label{subfig:GraphB}
 \end{subfigure}
 \hfill
 \begin{subfigure}[t]{0.3\textwidth}
  \centering
  \includegraphics[height=0.15\textheight]{Figures/GraphC.pdf}
  \caption{}
  \label{subfig:GraphC}
 \end{subfigure}
 \caption{Τρία Γραφήματα.}
 \label{fig:ThreeGraphs}
\end{figure}
\end{verbatim}

\begin{figure}[t]
	\centering
	\begin{subfigure}[t]{0.3\textwidth}
		\centering
		\includegraphics[height=0.15\textheight]{Figures/GraphA.pdf}
		\caption{}
		\label{subfig:GraphA}
	\end{subfigure}
	\hfill
	\begin{subfigure}[t]{0.3\textwidth}
		\centering
		\includegraphics[height=0.15\textheight]{Figures/GraphB.pdf}
		\caption{}
		\label{subfig:GraphB}
	\end{subfigure}
	\hfill
	\begin{subfigure}[t]{0.3\textwidth}
		\centering
		\includegraphics[height=0.15\textheight]{Figures/GraphC.pdf}
		\caption{}
		\label{subfig:GraphC}
	\end{subfigure}
	\caption{Τρία Γραφήματα.}
	\label{fig:ThreeGraphs}
\end{figure}

Με τις παραπάνω εντολές εισάγαμε στο Σχήμα~\ref{fig:ThreeGraphs} τρία 
υποσχήματα, στα οποία μπορούμε να αναφερθούμε και ξεχωριστά αν θέλουμε.
Στο Latex χρησιμοποιώντας τις αντίστοιχες ετικέτες που τους αναθέσαμε.
Στο Word χρησιμοποιείται αόρατος πίνακας με την δεύτερη γραμμή να περιέχει το κείμενο τον υπολεζάντων (τα α, β, γ, στην περίπτωση αυτή).
Οι αναφορές σε υποσχήμα γίνεται με κανονική αναφορά στο σχήμα και προσθήκη με το χέρι του α ή του β, κλπ. 


\section{Πίνακες}
\label{sec:Tables}
Με παρόμοιες εντολές μπορούμε να εισάγουμε και πίνακες.
Παρατηρήστε ότι η λεζάντα τοποθετείται πρίν (επάνω) από τον πίνακα.
Για παράδειγμα, με τις παρακάτω εντολές δημιουργούμε τον Πίνακα~\ref{tab:Example}.

\begin{verbatim}
\begin{table}[t]
 \centering
 \caption{Ένας Πίνακας.}
 \label{tab:Example}
 \begin{tabular}{| l || l | l | l |}
  \hline
  κελί 1 & κελί 2 & κελί 3 & κελί 4\\
  \hline
  \hline
  κελί 5 & κελί 6 & κελί 7 & κελί 8\\
  \hline
  κελί 9 & κελί 10 & κελί 11 & κελί 12\\
  \hline
  κελί 13 & κελί 14 & κελί 15 & κελί 16\\
  \hline
  κελί 17 & κελί 18 & κελί 19 & κελί 20\\
  \hline
 \end{tabular}
\end{table}
\end{verbatim}

\begin{table}[t]
	\centering
	\caption{Ένας Πίνακας.}
	\label{tab:Example}
	\begin{tabular}{| l || l | l | l |}
		\hline
		κελί 1 & κελί 2 & κελί 3 & κελί 4\\
		\hline
		\hline
		κελί 5 & κελί 6 & κελί 7 & κελί 8\\
		\hline
		κελί 9 & κελί 10 & κελί 11 & κελί 12\\
		\hline
		κελί 13 & κελί 14 & κελί 15 & κελί 16\\
		\hline
		κελί 17 & κελί 18 & κελί 19 & κελί 20\\
		\hline
	\end{tabular}
\end{table}


\section{Αλγόριθμοι}
\label{sec:Algorithms}
Για τη στοιχειοθεσία αλγορίθμων σε μορφή ψευδοκώδικα, όπως φαίνεται στον 
Αλγόριθμο~\ref{alg:Example} για παράδειγμα, χρησιμοποιούμε τις παρακάτω εντολές:

\begin{verbatim}
\begin{algorithm}[t]
 \caption{Υπολογισμός $y = x^n$.}
 \label{alg:Example}
 \begin{algorithmic}[1]
  \REQUIRE $n \geq 0 \vee x \neq 0$
  \ENSURE $y = x^n$
  \STATE $y \leftarrow 1$
  \IF{$n < 0$}
  \STATE $X \leftarrow 1 / x$
  \STATE $N \leftarrow -n$
  \ELSE
  \STATE $X \leftarrow x$
  \STATE $N \leftarrow n$
  \ENDIF
  \WHILE{$N \neq 0$}
  \IF{$N$ is even}
  \STATE $X \leftarrow X \times X$
  \STATE $N \leftarrow N / 2$
  \ELSE[$N$ is odd]
  \STATE $y \leftarrow y \times X$
  \STATE $N \leftarrow N - 1$
  \ENDIF
  \ENDWHILE
 \end{algorithmic}
\end{algorithm}
\end{verbatim}

\begin{algorithm}[t]
	\caption{Υπολογισμός $y = x^n$.}
	\label{alg:Example}
	\begin{algorithmic}[1]
		\REQUIRE $n \geq 0 \vee x \neq 0$
		\ENSURE $y = x^n$
		\STATE $y \leftarrow 1$
		\IF{$n < 0$}
		\STATE $X \leftarrow 1 / x$
		\STATE $N \leftarrow -n$
		\ELSE
		\STATE $X \leftarrow x$
		\STATE $N \leftarrow n$
		\ENDIF
		\WHILE{$N \neq 0$}
		\IF{$N$ is even}
		\STATE $X \leftarrow X \times X$
		\STATE $N \leftarrow N / 2$
		\ELSE[$N$ is odd]
		\STATE $y \leftarrow y \times X$
		\STATE $N \leftarrow N - 1$
		\ENDIF
		\ENDWHILE
	\end{algorithmic}
\end{algorithm}


\section{Μαθηματικά}
\label{sec:Mathematics}

Οι μαθηματικές εκφράσεις απαιτούν ειδική μορφοποίηση.
Δεν επιτρέπεται η χρήση απλού κειμένου.
Για τη στοιχειοθεσία μαθηματικών εκφράσεων σε Latex χρησιμοποιούμε
τα παρακάτω περιβάλλοντα:
\begin{itemize}
	\item Εξίσωση: \verb|\begin{equation} ... \end{equation}|.

	\begin{equation}
		S_{n} = 1 + \sum_{k=1}^{n}\frac{1}{k^{2} + k}\,,\quad n \in \mathbb{N}\,.
		\label{eq:Example}
	\end{equation}

	\item Θεώρημα: \verb|\begin{theorem} ... \end{theorem}|.

	\begin{theorem}
		Τhe square of the hypotenuse (the side opposite the right angle) is equal 
		to the sum of the squares of the other two sides.
	\end{theorem}

	\item Λήμμα: \verb|\begin{lemma} ... \end{lemma}|.

	\begin{lemma}
		If a prime divides the product of two numbers, it must divide at least 
		one of those numbers.
	\end{lemma}

	\item Πόρισμα: \verb|\begin{corollary} ... \end{corollary}|.

	\begin{corollary}
		In any right triangle, the hypotenuse is greater than any one of the 
		other sides, but less than their sum.
	\end{corollary}

	\item Γεγονός: \verb|\begin{fact} ... \end{fact}|.

	\begin{fact}
		It takes 8 minutes 17 seconds for light to travel from the Sun’s surface 
		to the Earth.
	\end{fact}

	\item Σημείωση: \verb|\begin{remark} ... \end{remark}|.

	\begin{remark}
		This is a remark.
	\end{remark}

	\item Ορισμός: \verb|\begin{definition} ... \end{definition}|.

	\begin{definition}
		Addition is bringing two or more numbers (or things) together to make 
		a new total.
	\end{definition}

	\item Παρατήρηση: \verb|\begin{observation} ... \end{observation}|.

	\begin{observation}
		This is an observation.
	\end{observation}

	\item Aπόδειξη: \verb|\begin{proof} ... \end{proof}|.

	\begin{theorem}[Fermat's Last Theorem]
		There are no positive integers $x$, $y$, and $z$ that satisfy the 
		equation $x^{n} + y^{n} = z^{n}$ for any integer value of $n > 2$.
	\end{theorem}
	\begin{proof}
		``I have discovered a truly marvellous proof of this, which this margin 
		is too narrow to contain.''
	\end{proof}
\end{itemize}

Οι λέξεις Θεώρημα, Λήμμα, κλπ. και ο αριθμός τους είναι με έντονα γράμματα και βρίσκονται σε εσοχή 30 στιγμών από το αριστερό περιθώριο του κειμένου.
Το κείμενο του θεωρήματος, λήμματος, κλπ γράφεται με πλάγια γράμματα και ξεκινούν από την ίδια γραμμή, όχι από κάτω.
Η απόδειξη ξεκινάει με την λέξη «Απόδειξη.» σε πλάγια γράμματα, ακολουθεί το κείμενο της απόδειξης (ξεκινώντας από την ίδια γραμμή) και στο τέλος υπάρχει ένα άδειο τετραγωνάκι, στοιχισμένο δεξιά.

Τα παραπάνω είναι αριθμημένα με παρόμοιο τρόπο όπως τα σχήματα: καινούρια αρίθμηση ανά κεφάλαιο και ο αριθμός τους περιλαμβάνει αριθμό κεφαλαίου, τελεία και αύξοντα αριθμό αντικειμένου μέσα στο κεφάλαιο.
Μόνο οι εξισώσεις που δεν αναφέρονται πουθενά στο κείμενο, επιτρέπεται να μην έχουν αρίθμηση.
Οι υπόλοιπες εξισώσεις αριθμούνται, αλλά η αρίθμηση φαίνεται στα δεξιά της εξίσωσης και είναι μέσα σε παρενθέσεις.
Η αναφορά σε εξίσωση γίνεται ως εξής: στην Εξίσωση~(\ref{eq:Example}) φαίνεται $\ldots$

Στο πρότυπο Word χρησιμοποιείται το στύλ Math για το κυρίως κείμενο που έχει την κατάλληλη μορφοποίηση.
Αλλά η αρίθμηση των θεωρημάτων κλπ, γίνεται χρησιμοποιώντας captions με ξεχωριστά labels ανά είδος.
Mετά το field code ενσωματώνεται μέσα στο κείμενο χωρίς να χρησιμοποιείται το στυλ caption.

\section{Περιεχόμενα και Κατάλογοι}
\label{subsec:Contents}

 Τα περιεχόμενα περιλαμβάνουν όλους τους καταλόγους σχημάτων, πινάκων, αλγορίθμων, το γλωσσάρι, την περίληψη, το Extended Abstract, τα κεφάλαια με τα δύο πρώτα επίπεδα ενοτήτων τους, τη Βιβλιογραφία, τα παραρτήματα και το ευρετήριο.
 Ο αριθμός σελίδας όπου ξεκινάει το αντίστοιχο τμήμα βρίσκεται στοιχισμένος δεξιά.
Για τις ενότητες μόνο, το κενό από το τέλος του τίτλου της ενότητας μέχρι τον αριθμό σελίδας καλύπτεται από τελείες, και δεν χρησιμοποιούνται έντονα γράμματα και αριθμοί.
 Επιπλέον οι ενότητες εμφανίζονται με εσοχή ώστε ο αριθμός ενότητας - υποενότητας να είναι ευθυγραμισμένος με την αρχή του τίτλου της προηγούμενης ενότητας - κεφαλαίου.
 Αντίθετα, για τα υπόλοιπα τμήματα, χρησιμοποιούνται έντονα γράμματα-αριθμοί και δεν υπάρχουν τελείες μεταξύ του τίτλου και του αριθμού σελίδας.

 Οι καταχωρήσεις των καταλόγων σχημάτων, πινάκων και αλγορίθμων μορφοποιούνται όπως οι ενότητες του κειμένου.

 Στο Word χρησιμοποιείται το στυλ Page Heading για τα τμήματα που δεν καταχωρούνται στα περιεχόμενα (αφιέρωση, ευχαριστίες, δημοσιεύσεις συγγραφέα, σύντομο βιογραφικό), ενώ για τα υπόλοιπα τμήματα, πλην κεφαλαίων και παραρτημάτων, υπάρχει το στυλ Page Heading TOC.
 Ακόμα και με τα παραπάνω, οι κατάλογοι που δημιουργούνται αυτόματα δεν είναι απόλυτα σύμφωνοι με το πρότυπο και χρειάζονται χειροκίνητες αλλαγές: διαγραφή των λέξεων \texttt{κεφάλαιο}, \texttt{παράρτημα}, \texttt{σχήμα}, \texttt{αλγόριθμος}, αλλαγή των λατινικών γραμμάτων των παραρτημάτων σε Ελληνικά, κ.α.




\section{Διαχείριση Βιβλιογραφίας}
\label{sec:Bibliography}

Το στυλ που ακολουθεί το πρότυπο είναι αυτό της IEEE Transactions.
Με το Latex η μορφοποίηση γίνεται αυτόματα.
Διαφορετικά θα χρειαστεί να ανατρέξετε στην περιγραφή που δίνει η IEEE\footnote{\url{https://ieee-dataport.org/sites/default/files/analysis/27/IEEE\%20Citation\%20Guidelines.pdf}}.
για πληροφορίες για το πως μορφοποιείται κάθε είδος αναφοράς.

Για τη δημιουργία της βιβλιογραφίας στο Latex χρησιμοποιούμε το πακέτο BibTeX.
Για αυτό απαιτείται μία βιβλιογραφική βάση δεδομένων, η οποία αποθηκεύεται ως 
ένα απλό αρχείο κειμένου με κατάληξη bib.
Το αρχείο αυτό περιέχει καταχωρήσεις της παρακάτω μορφής:

\begin{verbatim}
@article{Newman2003a,
 author = {Newman, Mark E. J.},
 title = {The Structure and Function of Complex Networks},
 journal = {SIAM Review},
 volume = {45},
 number = {2},
 pages = {167--256},
 year = {2003},
 doi = {10.1137/S003614450342480}
}
\end{verbatim}

Κάθε καταχώρηση ξεκινά με τη δήλωση του τύπου της αναφοράς. Το παραπάνω 
παράδειγμα αποτελεί αναφορά σε ένα άρθρο περιοδικού, επομένως η καταχώρηση 
ξεκινά με τη δήλωση \verb|@article|. Στη συνέχεια αναθέτουμε ένα μοναδικό 
κλειδί στην καταχώρηση, π.χ. \verb|Newman2003a|, το οποίο χρησιμοποιούμε 
στο κείμενο της διατριβής για να αναφερθούμε σε αυτή με την εντολή 
\verb|\cite{Newman2003a}|. Τέλος, συμπληρώνουμε τα πεδία του αντίστοιχου 
τύπου αναφοράς, μερικά από τα οποία είναι υποχρεωτικά. Για παράδειγμα, στις 
καταχωρήσεις άρθρων είναι υποχρεωτική η συμπλήρωση των πεδίων \verb|author|, 
\verb|title|, \verb|journal|, και \verb|year|.

Υπάρχουν αρκετές εφαρμογές διαχείρισης βιβλιογραφίας που συνεργάζονται με το Word, με γνωστότερα το Zotero, Mendeley και Endnote.
Η χρήση κάποιου από αυτά συνιστάται θερμά γιατί το να γίνει χειροκίνητα είναι πολύ επίπονο και χρονοβόρο

Η βιβλιογραφία της διατριβής στοιχειοθετείται αυτόματα μετά το τέλος των 
κεφαλαίων, με κάθε καταχώρηση να έχει έναν χαρακτηριστικό αριθμό.
Ο χαρακτηριστικός αριθμός της κάθε καταχώρησης εμφανίζεται μεταξύ αγκυλών 
στα σημεία του κειμένου της διατριβής όπου αναφερθήκαμε σε αυτή την καταχώρηση.
Για παράδειγμα, σε αυτήν την πρόταση αναφερόμαστε σε ένα άρθρο 
περιοδικού~\cite{Newman2003a}, σε μία εργασία συνεδρίου~\cite{DeCandia2007a}, 
σε μία τεχνική αναφορά~\cite{Jain1984a}, σε ένα βιβλίο~\cite{Golumbic2004a}, σε ένα διδακτορικό~\cite{beckmann_managing_2006}, σε ένα τμήμα βιβλίου~\cite{lai_efficient_2004} και σε μια ιστοσελίδα~\cite{rubino_ie9_nodate}.


