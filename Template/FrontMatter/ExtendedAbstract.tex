\chapter*{\cseextabstract}
% Εισαγωγή του κεφάλαιου στα περιεχόμενα
\addstarredchapter{\cseextabstract} % minitoc
% Για την χρήση των \@author, \@date και \@title
\makeatletter

\colorbox{gray}{Όνομα Επώνυμο}.\\ % Συμπληρώστε το όνομα σας (στη σωστή γλώσσα)
\cseextabstracttype\ \cseextabstractcs, \@date.\\
\cseextabstractdpt.\\
\colorbox{gray}{Τίτλος Διατριβής}.\\ % Συμπληρώστε τον τίτλο (στη σωστή γλώσσα)
\cseextabstractsup: \colorbox{gray}{Όνομα Επώνυμο}. % Συμπληρώστε το όνομα του καθηγητή σας

\makeatother
\bigskip
\bigskip

\noindent Εκτεταμένη περίληψη της εργασίας στην αντίθετη
γλώσσα από αυτήν του κειμένου. (Αν το κείμενο είναι στα
Ελληνικά αυτή η σελίδα πρέπει να είναι στα Αγγλικά, αν
το κείμενο είναι στα Αγγλικά τότε και αυτή η σελίδα να
είναι στα Ελληνικά.)

\y\noindent Προτεινόμενο μέγεθος: 2 σελίδες.

\y\noindent Μέγιστο μέγεθος: 4 σελίδες.